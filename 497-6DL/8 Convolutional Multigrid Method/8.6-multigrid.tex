
\section{Finite element method and convolution}\label{sec:mg}
Let us first briefly describe finite
element methods for the numerical solution of the following boundary
value problem
\begin{equation}
\label{laplace}
-\Delta u = f,  \mbox{ in } \Omega,\quad
u=0  \mbox{ on } \partial\Omega,\quad
\Omega=(0,1)^2.
\end{equation}
For the $x$ direction and the $y$ direction, we consider the partition:
\begin{equation}\label{partitionyx}
 0=x_0<x_1<\cdots<x_{n+1}=1, \quad x_i=\frac{i}{m+1},\quad (i=0,\cdots,m+1);
 \end{equation}
 \begin{equation}\label{partitiony}
 0=y_0<y_1<\cdots<y_{n+1}=1, \quad y_j=\frac{j}{n+1},\quad (j=0,\cdots,n+1).
\end{equation}
For $\Omega=(0,1)^2$, we just choose $m=n$. Such a uniform partition in the $x$ and $y$ directions leads us to a special example in two dimensions, a uniform square mesh $\R_h^2 = \big\{(ih,jh); i, j \in \Z\big\}$ (Figure \ref{fig:2dpartition}). 


\begin{figure}
\begin{center}
\setlength{\unitlength}{0.5mm}
\begin{picture}(45,45)(50,0)
\linethickness{0.25mm}
\multiput(0,0)(10,0){6}{\line(0,1){50}}
\multiput(0,0)(0,10){6}{\line(1,0){50}}
\put(0,0){\line(1,1){50}}
\put(10,0){\line(1,1){40}}
\put(20,0){\line(1,1){30}}
\put(30,0){\line(1,1){20}}
\put(40,0){\line(1,1){10}}
\put(0,10){\line(1,1){40}}
\put(0,20){\line(1,1){30}}
\put(0,30){\line(1,1){20}}
\put(0,40){\line(1,1){10}}
\put(47,34){$\displaystyle \left. \begin{array}{l}~ \\ ~\end{array}
\right\} h={1\over n+1}$}
\put(54,14){$\displaystyle N = n^2$}
\multiput(100,0)(10,0){6}{\line(0,1){50}}
\multiput(100,0)(0,10){6}{\line(1,0){50}}
\put(147,34){$\displaystyle \left. \begin{array}{l}~ \\ ~\end{array}
\right\} h={1\over n+1}$}
\put(154,14){$\displaystyle N = n^2$}
\end{picture}
\setlength{\unitlength}{0.5mm}
\end{center}
\label{fig:2dpartition}
\caption{Two-dimensional uniform grids for finite element}
\end{figure}

We consider two finite elements: continuous linear element and
bilinear element. These two finite element methods find $u_h\in V_h$
such that
\begin{equation}\label{Discrete:2d}
(\nabla u_h, \nabla v_h)=(f, v_h),\ \forall v_h\in V_h.
\end{equation}


Basis functions $\phi_{ij}$ satisfy 
\begin{equation}
  \label{NodalBasis}
\phi_{ij}(x_k,y_l)=\delta_{(i,j),(k,l)}.  
\end{equation}

Consider continuous linear finite element discretization of \eqref{laplace} on
the left triangulation in Fig \ref{fig:2dpartition}. The discrete
space for linear finite element is
$$
\mathcal V_h=\{v_h: v_h|_K\in P_1(K) \text{ and } v_h \text{ is globally continuous}\}.
$$ 

By simple computation, we have
\begin{equation}
(\nabla \phi_{i,j} , \nabla \phi_{k,l})=
\left\{
		\begin{array}{ll}
		-1 & \mbox{ if } |k-i|+|j-l|= 1,\\     
		4 & \mbox{ if } (k,l)=(i,j),\\
		0& \mbox{ elsewhere}.
		\end{array}
		\right.
\end{equation}
It is easy to verify that the formulation for the linear element method is 
\begin{equation}
  \label{2d-fe0}
A\ast u=4u_{i,j}-(u_{i+1,j}+u_{i-1,j}+u_{i,j+1}+u_{i,j-1})=f_{i,j},~~u_{i,j}=0~~\hbox{if}~~i ~~\hbox{or}~~ j\in \{0, n+1\},
\end{equation}
where 
\begin{equation}\label{fij_fe}
f_{i,j} = \int_{\Omega} f(x,y)\phi_{i,j}(x,y) {\rm d}x {\rm d}y \approx h^2 f(x_i, y_j).
\end{equation} 

\begin{definition}\label{def:convolution}
A convolution defined on $\mathbb{R}^{m\times n}$ is a linear mapping 
$K\ast: \mathbb{R}^{m\times n}\mapsto \mathbb{R}^{m\times n}$ defined with padding,  
for any $g \in \mathbb{R}^{m\times n}$ by:
%We first consider $\theta$ a convolution operator (with stride $1$) 
%and padding:
\begin{equation}\label{con010}
[K \ast g]_{i,j} = \sum_{p,q=-k}^k K_{p, q} g_{i + p, j + q}, \quad i=1:m, j = 1:n.
\end{equation}
\end{definition}
The coefficients in \eqref{con010} constitute  a kernel matrix
\begin{equation}
K \in \mathbb{R}^{(2k+1) \times (2k+1)},
\end{equation}
where $k$ is often taken as a small integer. 
Here we note that the indices for the entries in $K$ are given in a special way. 
For example, if $k=1, K\in \mathbb R^{3\times 3}$, and 
$$
K=\begin{pmatrix}
	K_{-1,-1} &K_{-1,0} &K_{-1,1} \\
	K_{0,-1} &K_{0,0} &K_{0,1} \\
	K_{1,-1} &K_{1,0} &K_{1,1} \\
	\end{pmatrix},
$$
for we may have the following 2D Laplacian kernel
\begin{equation}\label{key}
K=\begin{pmatrix}
0 &-1 &0\\
-1 &4&-1 \\
0 &-1 &0 \\
\end{pmatrix}.
\end{equation}  
Here padding means how $ g_{i+ p, j + q}$ is defined
when $(i+ p, j + q)$ is out of $1:m$ or $1:n$. 
The following three choices are often used
\begin{equation}\label{eq:padding}
g_{i + p, j + q} = \begin{cases}
0,  \quad &\text{zero padding}, \\
f_{(i + p)\pmod{m}, (s + q)\pmod{n}},  \quad &\text{periodic padding}, \\
f_{|i-1 +p|, |j -1  +q|},  \quad &\text{reflected padding}, \\
\end{cases}
\end{equation}
if 
\begin{equation}
i + p \notin \{1, 2, \dots, m\} ~\text{or} ~  j+ q \notin \{1, 2, \dots, n\}.
\end{equation}
Here $ d \pmod{m} \in \{1, \cdots, m\} $  means the remainder when $d$ is divided by $m$.

\begin{definition}\label{def:convolution2}
For $g \in \mathbb{R}^{m\times n}$, convolution with stride $2$ is defined as 
\begin{equation}\label{stride_2}
[K \ast_2 g]_{i,j} = \sum_{p,q=-k}^k K_{p,q} g_{2i + p-1, 2j + q-1},  
\quad i = 1: \lfloor \frac{m+1}{2}\rfloor , j = 1: \lfloor \frac{n+1}{2} \rfloor.
\end{equation}
\end{definition}

Using the convolutional notation \eqref{con010}, \eqref{2d-fe0} can be written as 
\begin{equation}
  \label{eq:Ac}
A\ast u=f
\end{equation}
with 
\begin{equation}
  \label{Ac}
A=
\begin{pmatrix}
0&-1&0\\
-1&4&-1\\
0&-1&0
\end{pmatrix}
.
\end{equation}

\begin{proposition}\label{prop:A}
The mapping $A\ast$ has following properties
\begin{enumerate}
\item $A$ is symmetric, namely 
$$
(A\ast u, v)_{l^2}=(u,A\ast v)_{l^2}.
$$
\item  $(A\ast v, v)_F>0, ~~~\hbox{if}~~~v\neq 0.$
\item $A\ast u=f$ if and only if 
\begin{equation}\label{minProblem}
u\in \argmin_{v\in \mathcal V_h} J(v)={1\over 2}(A\ast v,v)-(f,v).
\end{equation}
\item The eigenvalues $\lambda_{kl}$ and eigenvectors $u^{kl}$ of $A$ are given by
$$
\lambda_{kl}=4(\sin^2\frac{k\pi}{2(n+1)}+ \sin^2\frac{l\pi}{2(n+1)}),
$$
$$
u_{ij}^{kl}=\sin \frac{ki\pi}{n+1}\sin \frac{lj\pi}{n+1},\ 1\leq i\leq n,\ 1\leq j\leq n,
$$
and  $\rho(A)<8$. Furthermore,
\begin{equation*}
\lambda_{n,n}=8\cos^2\frac{\pi}{2(n+1)}\approx 8(1- ({\pi\over 2(n+1)})^2) \approx 8-{2\pi^2\over (n+1)^2}
\end{equation*}
\end{enumerate}
\end{proposition}


  




