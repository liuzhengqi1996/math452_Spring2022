
\section{Weierstrass Theorem}  
To approximate any continuous function, a very simple idea is to approximate the function in a polynomial space. 
An important property of this space is that polynomials can approximate any reasonable function!
\begin{itemize}
\item $P_n(\mathbb{R}^d)$ is dense in $C(\Omega)$ [Weierstrass theorem]
\item $P_n(\mathbb{R}^d)$ is dense in all Sobolev spaces: $L^2(\Omega), W^{m,p}(\Omega), \ldots$
\end{itemize}

\input{6DL/WTheoremProof}
 
\subsection{Curse of dimensionality}
Number of coefficients for polynomial space $P_n(\mathbb{R}^d)$  is
$$
	N = \binom{d+n}{n} = \frac{(n+d)!}{d!n!}.
$$
For example $n = 100$:
		\begin{table}
			\centering
			\begin{tabular}{|c|c|c|c|}
				\hline
				$d = $&  $2$ &  $4$ & $8$\\
				\hline
				$N=$  & $5\times10^3$ & $4.6\times10^6$  & $3.5\times10^{11}$ 	\\
				\hline
			\end{tabular}
		\end{table}
As the this table shows, the dimension of the polynomial space $P_n(\mathbb{R}^d)$   increases rapidly as the degree $n$ increases. This leads to an extremely large space therefore very expensive to approximate functions in polynomial spaces in high dimensions.
\input{6DL/Runge}
\input{6DL/Fourier}
\input{6DL/DNN_Qualitative}
